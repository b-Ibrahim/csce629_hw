% Comment lines start with %
% LaTeX commands start with \

\documentclass[12pt]{article}  % This is an article with font size 12-point

% Packages add features
\usepackage{times}     % font choice
\usepackage{amsmath}   % American Mathematical Association math formatting
\usepackage{amsthm}    % nice formatting of theorems
\usepackage{latexsym}  % provides some more symbols
\usepackage{fullpage}  % uses most of the page (1-inch margins)

\setlength{\parskip}{.1in}  % increase the space between paragraphs

\renewcommand{\baselinestretch}{1.1}  % increase the space between lines

% Convenient renaming of symbols for logic formulas
\newcommand{\NOT}{\neg}
\newcommand{\AND}{\wedge}
\newcommand{\OR}{\vee}
\newcommand{\XOR}{\oplus}
\newcommand{\IMPLIES}{\rightarrow}
\newcommand{\IFF}{\leftrightarrow}

% Actual content starts here.
\begin{document}

\begin{center}         % center all the material between begin and end
{\large                % use larger font
CSCE 629-601, Analysis of Algorithms \\  % \\ is line break
Spring 2017 \\
Homework 1 \\
Bassem Ibrahim}
\end{center}

% blank line separates paragraphs.  First line of a paragraph is automatically
% indented.  

\rule{6in}{.1pt}       % horizontal line 6 inches long and .1 point high
                    
\noindent              % don't indent
{\bf Instructions:}    % \bf makes text boldface
                       % \em makes text emphasized (italics)

\begin{itemize}        % makes an itemized list
\item The numbered exercises and problems are from the textbook.  
\item Each exercise is worth 10 points.
\item Grading will be based on correctness and clarity.
\item Always justify your answers, even if the question does not
      explicitly say so.  Argue correctness and running time of all algorithms.
\item Don't forget to acknowledge all sources of assistance on the cover
      sheet, and write up your solutions on your own.
\item {\em Turn in a paper copy of the .pdf file with a
      filled-in assignment cover sheet at the beginning of class on 
      Wednesday, Feb 1, 2017.}
\end{itemize}

\rule{6in}{.1pt}       % horizontal line 6 inches long and .1 point high

\noindent
{\bf LaTeX hints:}  Read this .tex file for some explanations that are in
the comments.

Math formulas are enclosed in \$ signs, e.g., {\tt \$x + y = z\$}
becomes $x + y = z$.

Logical operators: $\NOT, \AND, \OR, \XOR, \IMPLIES, \IFF$.

Here is a truth table using the ``tabular'' environment:

\begin{center}
\begin{tabular}{|c|c|}  % two columns, both centered (c), 
                        % divided by vertical lines (|)
\hline                  % horizontal line
$p$ & $\NOT p$ \\       % separate column entries with &
\hline
\hline
T & F \\
\hline
F & T \\
\hline
\end{tabular}
\end{center}

{\bf ** Delete the instructions and the LaTeX hints in your solution. **}

\rule{6in}{.1pt}       % horizontal line 6 inches long and .1 point high

%---------------------------------------------------------------------

\noindent
{\bf Exercise 2.3-3:}
** YOUR ANSWER GOES HERE **

\rule{6in}{.1pt}       % horizontal line 6 inches long and .1 point high

\noindent
{\bf Exercise 2.3-7:}
** YOUR ANSWER GOES HERE **

\rule{6in}{.1pt}       % horizontal line 6 inches long and .1 point high

\noindent
{\bf Exercise 3.1-2:}  Prove $(n+a)^b= \Theta(n^b)$

$c_1.(n^b)\leq(n+a)^b\leq c_2.(n^b)$

Divide by $n^b$

$c_1\leq(\frac{n+a}{n})^b\leq c_2$

For $n\geq 1$, $c_1\leq (1+a)^b$ and $c_2\geq 1$


\rule{6in}{.1pt}       % horizontal line 6 inches long and .1 point high

\noindent
{\bf Exercise 3.1-4:}

Is $2^{n+1}=O(2^n)$?

$2^{n+1}\leq c.(2^n)$

Divide by $2^n$

$2^{1}\leq c$

$2^{n+1}=O(2^n)$ if $c\geq 2$ $\forall$ $n\geq 1$

Is $2^{2n}=O(2^n)$?

$2^{2n}\leq c.(2^n)$

Divide by $2^n$

$2^{n}\leq c$

$2^{n+1}\neq O(2^n)$ if because $c$ is function of $n$

\rule{6in}{.1pt}       % horizontal line 6 inches long and .1 point high

\noindent
{\bf Exercise 3.2-3:}
** YOUR ANSWER GOES HERE **

\rule{6in}{.1pt}       % horizontal line 6 inches long and .1 point high

\noindent
{\bf Exercise 4.5-3:}
** YOUR ANSWER GOES HERE **

\rule{6in}{.1pt}       % horizontal line 6 inches long and .1 point high

\noindent
{\bf Problem 4-3, parts (a) through (e):}
** YOUR ANSWER GOES HERE **

\rule{6in}{.1pt}       % horizontal line 6 inches long and .1 point high

\noindent
{\bf Problem 4-5:}  This one is just for fun and won't be graded.
There was a typo in old printings of the book:  For part (a), assume
that {\em at least} $n/2$ chips are bad.
Here is a suggestion for how to start with part (a).  Suppose $n$ is
even and exactly $n/2$ of the chips are bad.
Show that even if we test all pairs of chips, for a total of $n(n-1)$ tests,
there are two different scenarios (choices for which chips are good
and which are bad) in which the outputs of the chips are the same.
As a warm-up, consider the case of $n=2$, where in one scenario
chip 1 is good and chip 2 is bad, while in the second scenario
chip 1 is bad and chip 2 is good.  In this case, there is only one test
to be done (chips 1 and 2 together).  Find an output for chip 1 and
an output for chip 2 that are possible in both the scenarios.

** YOUR ANSWER GOES HERE **

\rule{6in}{.1pt}       % horizontal line 6 inches long and .1 point high

\noindent
{\bf *** there will be a few more ***}

\end{document}
